\documentclass{beamer}

\usepackage{geometry}
\usepackage{graphicx}
%\usepackage{wrapfig}
\usepackage{amsmath}

%\useoutertheme{infolines}
\usetheme{Boadilla}
\usecolortheme{seahorse}
\setbeamertemplate{navigation symbols}{}
\title{Counting Bits in an Array}
\newcommand{\shorttitle}{64 Bit Intel Assembly Language}
\newcommand{\shortauthor}{\copyright 2011 Ray Seyfarth}
\author{Ray Seyfarth}
\begin{document}


\usefoottemplate{\vbox{
\tinycolouredline{structure!55}%
 {\color{white}{\textbf{\shorttitle}\hfill\textbf{\shortauthor}}}%
}}

\begin{frame}
    \titlepage
\end{frame}

\begin{frame}
\frametitle{Outline}
\tableofcontents
\end{frame}

\section{Overview}

\begin{frame}
    \frametitle{Overview of counting bits}
    \begin{itemize}
        \item The basic goal is to count the number of 1 bits in an array
        \item Several solutions are examined in C and assembly
        \item In general C and assembly perform similarly
        \item On CPUs with the {\tt popcnt} instruction, assembly rules
    \end{itemize}
\end{frame}

\section{Counting 1 bits in C}

\begin{frame}[fragile]
    \frametitle{A simple C solution}
\small
\begin{verbatim}
    long popcnt_array ( long *a, int size )
    {
        int w, b;
        long word;
        long n;
        n = 0;
        for ( w = 0; w < size; w++ ) {
            word = a[w];
            n += word & 1;
            for ( b = 1; b < 64; b++ ) {
                n += (word >> b) & 1;
            }
        }
        return n;
    }
\end{verbatim}
    \begin{itemize}
        \item Checking every bit took 4.74 seconds to call {\tt popcnt\_array}
              1000 times with 100000 longs (64 bits)
    \end{itemize}
\end{frame}

\begin{frame}[fragile]
    \frametitle{Ending the loop earlier}
\small
    \begin{itemize}
        \item A slightly better algorithm ends the inner loop when {\tt word} = 0
        \item The time dropped to 3.34 seconds
    \end{itemize}
\begin{verbatim}
    long popcnt_array ( unsigned long *a, int size )
    {
        int w, b;
        unsigned long word;
        long n;
    
        n = 0;
        for ( w = 0; w < size; w++ ) {
            word = a[w];
            while ( word != 0 ) {
                n += word & 1;
                word >>= 1;
            }
        }
        return n;
    }
\end{verbatim}
\end{frame}

\section{Counting 1 bit in assembly}

\begin{frame}[fragile]
    \frametitle{Counting 1 bits in assembly}
    \begin{itemize}
        \item I unrolled the inner loop 64 times
        \item The code is too long to place in a slide
        \item I split each 64 bit word into 4 16 bit words in separate registers
        \item Then I added each bit of the four words into 4 different registers
              allowing out-of-execution, pipeline filling and parallelism
        \item It performed the test in 2.52 seconds, a bit better than C at 3.34
        \item I did have a function of 1123 bytes
    \end{itemize}
\end{frame}

\section{Precomputing 1 counts for all patterns of bytes}

\begin{frame}[fragile]
    \frametitle{Precomputing 1 counts for all pattern of bytes}
\small
\begin{verbatim}
    long popcnt_array ( long *a, int size )
    {
        int b;
        long n;
        int word;
    
        n = 0;
        for ( b = 0; b < size*8; b++ ) {
            word = ((unsigned char *)a)[b];
            n += count[word];
        }
        return n;
    }
\end{verbatim}
    \begin{itemize}
        \item The count array had a static initializer with 256 counts
        \item The time dropped to 0.24 seconds, 10 times faster than the last version
        \item I could only tie this code with an assembly version
    \end{itemize}
\end{frame}

\section{Using the popcnt instruction}

\begin{frame}[fragile]
    \frametitle{Using the {\tt popcnt} instruction}
    \begin{itemize}
        \item Some newer computers (Intel Core i series and some Opterons) have
              a {\tt popcnt} instruction which exactly matches the problem.`
        \item After unrolling the loop 2 times, the operation took 0.04 seconds
              on a Core i7 at 3.4 GHz
    \end{itemize}
\begin{verbatim}
.count_more:
        popcnt  rdx, [rdi+rcx*8]
        add     rax, rdx
        popcnt  r9, [rdi+rcx*8+8]
        add     r8, r9
        add     rcx, 2
        cmp     rcx, rsi
        jl      .count_more
        add     rax, r8
\end{verbatim}
\end{frame}

\end{document}

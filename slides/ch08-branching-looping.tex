\documentclass{beamer}

\usepackage{geometry}
\usepackage{graphicx}
%\usepackage{wrapfig}
\usepackage{amsmath}

%\useoutertheme{infolines}
\usetheme{Boadilla}
\usecolortheme{seahorse}
\setbeamertemplate{navigation symbols}{}
\title{Branching and Looping}
\newcommand{\shorttitle}{64 Bit Intel Assembly Language}
\newcommand{\shortauthor}{\copyright 2011 Ray Seyfarth}
\author{Ray Seyfarth}
\begin{document}


\usefoottemplate{\vbox{
\tinycolouredline{structure!55}%
 {\color{white}{\textbf{\shorttitle}\hfill\textbf{\shortauthor}}}%
}}

\begin{frame}
    \titlepage
\end{frame}

\begin{frame}
    \frametitle{Branching and looping}
    \begin{itemize}
        \item So far we have only written ``straight line'' code
        \item Conditional moves helped spice things up
        \item In addition conditional moves kept the pipeline full
        \item But conditional moves are not always faster than branching
        \item But we need loops to process each bit in a register
        \item Repeated code can be faster, but there is a limit
        \item In the next chapter we will work with arrays
        \item Here we will need to process differing amounts of data
        \item Repeated code is too inflexible
        \item We need loops
        \item To handle code structures like if/else we need both
              conditional and unconditional branch statements
    \end{itemize}
\end{frame}

\begin{frame}
\frametitle{Outline}
\tableofcontents
\end{frame}

\section{Unconditional jump}

\begin{frame}[fragile]
    \frametitle{Unconditional jump}
    \begin{itemize}
        \item An unconditional jump is equivalent to a {\tt goto}
        \item But jumps are necessary in assembly, while high level
              languages could exist without {\tt goto}
        \item The unconditional jump looks like\\
    {\tt jmp     label}
        \item The {\tt label} can be any label in the program's text segment
        \item Humans think of parts of the text segment as functions
        \item The computer will let you jump anywhere
        \item You can try to jump to a label in the data segment, which
              hopefully will fail
        \item The assembler will generate an instruction register ({\tt rip})
              relative location to jump
        \item The simplest form uses an 8 bit immediate: -128 to +127 bytes
        \item The next version is 32 bits: plus or minus 2 GB
        \item The short version takes up 2 bytes; the longer version 5 bytes
        \item The assembler figures this out for you
    \end{itemize}
\end{frame}

\begin{frame}
    \frametitle{Unconditional jumps can vary}
    \begin{itemize}
        \item An unconditional jump can jump to a location specified
              by a register's content or a memory location
        \item You could use a conditional move to hold either of 2 locations
              in a register and jump to the proper location
        \item It is simpler to just use a conditional jump
        \item However you can construct an efficient switch statement
              by expanding this idea
        \item You need an array of addresses and an index for the array
              to select which address to use for the jump
    \end{itemize}
\end{frame}

\begin{frame}[fragile]
    \frametitle{Unconditional jump used as a switch}
\small
\begin{verbatim}
        segment .data
switch: dq      main.case0
        dq      main.case1
        dq      main.case2
i:      dq      2
        segment .text
        global  main                ; tell linker about main
main:   mov     rax, [i]            ; move i to rax
        jmp     [switch+rax*8]      ; switch ( i )
.case0:
        mov     rbx, 100            ; go here if i == 0
        jmp     .end
.case1:
        mov     rbx, 101            ; go here if i == 1
        jmp     .end     
.case2:
        mov     rbx, 102            ; go here if i == 2
.end:   xor     eax, eax
        ret
\end{verbatim}
\end{frame}

\section{Conditional jump}

\begin{frame}[fragile]
    \frametitle{Conditional jump}
    \begin{itemize}
        \item First you need to execute an instruction which sets some flags
        \item Then you can use a conditional jump
        \item The general pattern is \\
    {\tt jCC    label}
        \item The {\tt CC} means a condition code
    \end{itemize}
\begin{center}
  \begin{tabular}{|l|l|l|l|}
    \hline
    instruction & meaning & aliases & flags \\
    \hline
    jz   & jump if zero & je & ZF=1 \\
    \hline
    jnz  & jump if not zero & jne & ZF=0 \\
    \hline
    jg   & jump if $>$ zero & jnle ja & ZF=0, SF=0 \\
    \hline
    jge  & jump if $\ge$ zero & jnl & SF=0 \\
    \hline
    jl   & jump if $<$ zero & jnge js & SF=1 \\
    \hline
    jle  & jump if $\le$ zero & jng & ZF=1 or SF=1 \\
    \hline
    jc   & jump if carry & jb jnae & CF=1 \\
    \hline
    jnc  & jump if not carry & jae jnb & CF=0 \\
    \hline
  \end{tabular}
\end{center}
\end{frame}

\begin{frame}[fragile]
    \frametitle{Simple if statement}
\begin{verbatim}
    if ( a < b ) {
        temp = a;
        a = b;
        b = temp;
    }
\end{verbatim}

\begin{verbatim}
        mov   rax, [a]
        mov   rbx, [b]
        cmp   rax, rbx
        jge   in_order
        mov   [temp], rax
        mov   [a], rbx
        mov   [b], rax
    in_order:
\end{verbatim}
\end{frame}

\begin{frame}[fragile]
    \frametitle{If statement with an else clause}
\begin{verbatim}
    if ( a < b ) {
        max = b;
    } else {
        max = a;
    }
\end{verbatim}

\begin{verbatim}
        mov   rax, [a]
        mov   rbx, [b]
        cmp   rax, rbx
        jnl   else
        mov   [max], rbx
        jmp   endif
else:   mov   [max], rax
endif:
\end{verbatim}
\end{frame}

\section{Looping with conditional jumps}

\begin{frame}[fragile]
    \frametitle{Looping with conditional jumps}
    \begin{itemize}
        \item You could construct any form of loop using conditional jumps
        \item We will model our code after C's loops
        \item {\tt while}, {\tt do ... while} and {\tt for}
        \item We will also consider {\tt break} and {\tt continue}
        \item {\tt break} and {\tt continue} can be avoided in C, though
              sometimes the result is less clear
        \item The same consideration applies for assembly loops as well
    \end{itemize}

\end{frame}

\begin{frame}[fragile]
    \frametitle{Counting 1 bits in a quad-word}
\begin{verbatim}
    sum = 0;
    i = 0;
    while ( i < 64 ) {
        sum += data & 1;
        data = data >> 1;
        i++;
    }
\end{verbatim}
    \begin{itemize}
        \item There are much faster ways to do this
        \item But this is easy to understand and convert to assembly
    \end{itemize}
\end{frame}

\begin{frame}[fragile]
    \frametitle{Counting 1 bits in a quad-word in assembly}
\small
\begin{verbatim}
        segment .text
        global  main
main:   mov     rax, [data] ; rax holds the data
        xor     ebx, ebx    ; clear since setc will fill in bl
        xor     ecx, ecx    ; i = 0;
        xor     edx, edx    ; sum = 0;
while:  cmp     rcx, 64     ; while ( i < 64 ) {
        jnl     end_while   ; requires testing on opposite
        bt      rax, 0      ; data & 1
        setc    bl          ; move result of test to bl
        add     edx, ebx    ; sum += data & 1;
        shr     rax, 1      ; data = data >> 1;
        inc     rcx         ; i++;
        jmp     while       ; end of the while loop
end_while:
        mov     [sum], rdx  ; save result in memory
        xor     eax, eax    ; return 0 from main
        ret
\end{verbatim}
\end{frame}
\begin{frame}[fragile]
    \frametitle{Code generated by {\tt gcc -O3 -S countbit.s}}
\begin{verbatim}
        movq    data(%rip), %rax
        movl    $64, %ecx
        xorl    %edx, %edx
.L2:
        movq    %rax, %rsi
        sarq    %rax
        andl    $1, %esi
        addq    %rsi, %rdx
        subl    $1, %ecx
        jne     .L2
\end{verbatim}
    \begin{itemize}
        \item AT\&T syntax: operands are reversed and names are more explicit
        \item The compiler counted down from 64
        \item Converted the loop to test at the bottom
        \item Loop has 2 fewer instructions
        \item Is it faster to use {\tt movq} and {\tt andl}?
    \end{itemize}
\end{frame}

\begin{frame}
    \frametitle{Learning from the compiler}
    \begin{itemize}
        \item The compiler writers know the instruction set very well
        \item Most likely {\tt movq} and {\tt andl} is faster
        \item Testing would tell if the other method is superior
        \item I also tried the compiler option ``{\tt -funroll-all-loops}''
        \item The compiler added up values for 8 bits in 1 loop iteration
        \item 8 bits in a 24 instruction loop vs 1 bit in a six instruction
              loop
        \item This makes it twice as fast, but the instructions use many
              different registers allowing parallel execution in 1 core
        \item Loop unrolling can help a lot with 16 registers
        \item Examining the generated code should mean than you do no worse
        \item Clever reorganization can beat the compiler
    \end{itemize}
\end{frame}

\begin{frame}[fragile]
    \frametitle{Do-while loops}
    \begin{itemize}
        \item Strict translation of a while loop uses 2 jumps
        \item It save a jump to the top if you use a {\tt do-while} loop
    \end{itemize}
\begin{verbatim}
    do {
        statements;
    } while ( condition );
\end{verbatim}
    \begin{itemize}
        \item A {\tt do-while} loop always executes the loop body at least
              once
        \item You can always place an {\tt if} statement around a 
              {\tt do-while} to make it behave like a {\tt while} loop
    \end{itemize}

\begin{verbatim}
    if ( condition ) {
        do {
            statements;
        } while ( condition );
    }
\end{verbatim}
    \begin{itemize}
        \item Don't do this in C - let the compiler do it for you
    \end{itemize}
\end{frame}

\begin{frame}[fragile]
    \frametitle{Ugly C code to search through an array}
\begin{verbatim}
    i = 0;
    c = data[i];
    if ( c != 0 ) do {
        if ( c == x ) break;
        i++;
        c = data[i];
    } while ( c != 0 );
    n = c == 0 ? -1 : i;
\end{verbatim}
\end{frame}

\begin{frame}[fragile]
    \frametitle{Assembly code to search through an array}
    \begin{verbatim}
        mov     bl, [x]        ; value being sought
        xor     ecx, ecx       ; i = 0;
        mov     al, [data+rcx] ; c = data[i]
        cmp     al, 0          ; if ( c != 0 ) {
        jz      end_while      ; skip loop for empty string
while:
        cmp     al, bl         ; if ( c == x ) break; 
        je      found
        inc     rcx            ; i++;
        mov     al, [data+rcx] ; c = data[i];
        cmp     al, 0          ; while ( c != 0 );
        jnz     while
end_while:
        mov     rcx, -1        ; If we get here, we failed
found:  mov     [n], rcx       ; Assign either -1 or the 
                               ; index where x was found
    \end{verbatim}

\end{frame}

\begin{frame}[fragile]
    \frametitle{Counting loops}
\begin{verbatim}
    for ( i = 0; i < n; i++ ) {
        c[i] = a[i] + b[i];
    }
\end{verbatim}
\begin{verbatim}
        mov     rdx, [n]       ; use rdx for n
        xor     ecx, ecx       ; i (rdx) = 0
for:    cmp     rcx, rdx       ; i < n
        je      end_for        ; get out if equal
        mov     rax, [a+rcx*8] ; get a[i]
        add     rax, [b+rcx*8] ; a[i] + b[i]
        mov     [c+rcx*8], rax ; c[i] = a[i] + b[i];
        inc     rcx            ; i++
        jmp     for            ; too bad, loop has 2 jumps
end_for:
\end{verbatim}
    \begin{itemize}
        \item We could use a test before the loop
        \item We could do loop unrolling
    \end{itemize}
\end{frame}

\section{Loop instructions}

\begin{frame}[fragile]
    \frametitle{Loop instructions}
    \begin{itemize}
        \item The CPU has instructions like {\tt loop} and {\tt loopne} which
              designed for loops
        \item They decrement {\tt rcx} and do the branch if {\tt rcx} is not 0
        \item It is faster to use {\tt dec} and {\tt jnz} instead
        \item The label must be within -128 to +127 bytes of {\tt rip}
        \item Probably pointless
    \end{itemize}
\begin{verbatim}
        mov    ecx, [n]
        sub    ecx, 1
more:   cmp    [data+rcx],al
        loopne more
        mov    [loc], ecx
\end{verbatim}
\end{frame}

\section{Repeat string (array) instructions}

\begin{frame}[fragile]
    \frametitle{Repeat string (array) instructions}
    \begin{itemize}
        \item The repeat instruction ({\tt rep}) works in conjunction
              with string (array) instructions
        \item You first set {\tt rcx} to be the number of repetitions
        \item You set {\tt rsi} to the address of source data
        \item And set {\tt rdi} to be the address of destination data
        \item Then you use a command like
    \end{itemize}
\begin{verbatim}
    rep     movsb
\end{verbatim}
    \begin{itemize}
        \item The previous command would copy an array of bytes
        \item Some string instructions include tests for early termination
        \item The string instructions can also be used without {\tt rep}
    \end{itemize}
\end{frame}

\begin{frame}[fragile]
    \frametitle{Store instruction}
    \begin{itemize}
        \item The {\tt stosb} instruction stores the byte in {\tt al}
              at the address specified in {\tt rdi} and increments {\tt rdi}
        \item If the direction flag is set it decrements {\tt rdi}
        \item There are also {\tt stosw}, {\tt stosd} and {\tt stosq}
              to operate 2, 4 and 8 byte quantities
    \end{itemize}
\begin{verbatim}
    mov     eax, 1
    mov     ecx, 1000000
    lea     rdi, [destination]
    rep     stosd      ; place 1000000 1's in destination
\end{verbatim}
\end{frame}

\begin{frame}[fragile]
    \frametitle{Store instruction}
    \begin{itemize}
        \item There are a collection of load string instructions
              which copy data from the address pointed at by {\tt rsi}
              and increment (or decrement) {\tt rsi}
        \item Using {\tt rep lodsb} seems pointless
        \item The code below uses {\tt lodsb} and optionally {\tt stosb}
              to copy none carriage return characters
    \end{itemize}
\begin{verbatim}
        lea     rsi, [source]
        lea     rdi, [destination]
        mov     ecx, 1000000   ; number of iterations
more:   lodsb                  ; get the next byte in al
        cmp     al, 13         ; if al is not 13 store al
        je      skip
        stosb                  ; store al in destination
skip:   sub     ecx, 1         ; count down
        jnz     more
\end{verbatim}
\end{frame}

\begin{frame}[fragile]
    \frametitle{Scan instruction}
    \begin{itemize}
        \item There are a collection of scan string instructions
              which scan data from the address pointed at by {\tt rsi}
              and increment (or decrement) {\tt rsi}
        \item They compare data against {\tt al}, {\tt ax}, ...
        \item Below is a version of the C {\tt strlen} function
    \end{itemize}
\begin{verbatim}
        segment .text
        global  strlen
strlen: cld                 ; prepare to increment rdi
        mov     rcx, -1     ; maximum number of iterations
        xor     al, al      ; will scan for 0
        repne   scasb       ; repeatedly scan for 0
        mov     rax, -2     ; start at -1, end 1 past the end
        sub     rax, rcx   
        ret
\end{verbatim}
\end{frame}


\begin{frame}[fragile]
    \frametitle{Compare instruction}
    \begin{itemize}
        \item The compare string instructions compare the data
              pointed at by {\tt rdi} and {\tt rsi}
        \item The code below implements the C {\tt memcmp} function
    \end{itemize}
\begin{verbatim}
        segment .text
        global  memcmp
memcmp: mov     rcx, rdx
        repe    cmpsb      ; compare until end or difference
        cmp     rcx, 0
        jz      equal      ; reached the end
        movzx   eax, byte [rdi-1]
        movsx   ecx, byte [rsi-1]
        sub     eax, ecx
        ret
equal:  xor     eax, eax
        ret
\end{verbatim}
\end{frame}

\begin{frame}
    \frametitle{Setting and clearing the direction flag}
    \begin{itemize}
        \item The string operations increment their addresses if the
              direction flag is 0
        \item They decrement their address is the direction flag is 1
        \item Use {\tt cld} to prepare for increasing addresses
        \item Use {\tt std} to prepare for decreasing addresses
        \item Functions are expected to leave the direction flag set to 0
    \end{itemize}
\end{frame}

\end{document}

\documentclass{beamer}

\usepackage{geometry}
\usepackage{graphicx}
%\usepackage{wrapfig}
\usepackage{amsmath}

%\useoutertheme{infolines}
\usetheme{Boadilla}
\usecolortheme{seahorse}
\setbeamertemplate{navigation symbols}{}
\title{Arrays}
\newcommand{\shorttitle}{64 Bit Intel Assembly Language}
\newcommand{\shortauthor}{\copyright 2011 Ray Seyfarth}
\author{Ray Seyfarth}
\begin{document}


\usefoottemplate{\vbox{
\tinycolouredline{structure!55}%
 {\color{white}{\textbf{\shorttitle}\hfill\textbf{\shortauthor}}}%
}}

\begin{frame}
    \titlepage
\end{frame}

\begin{frame}
    \frametitle{Arrays}
    \begin{itemize}
        \item An array is a contiguous collection of memory cells
              of a specific type
        \item The start address of an array is the address of the first element
        \item This is associated with the label given before a data definition
              in the data segment or a data reservation in the bss segment
        \item The first index of an array in C/C++ and assembly is 0
        \item Each subsequent array cell is at a higher memory address
        \item The final index for an array of $n$ elements is $n-1$
        \item Some high level languages use different or user-selectable starting
              indices for arrays
        \item Fortran defaults to 1
        \item 0 is the most logical first index because it simplifies array address
              computation
    \end{itemize}

\end{frame}

\begin{frame}
\frametitle{Outline}
\tableofcontents
\end{frame}

\section{Array address computation}

\begin{frame}[fragile]
    \frametitle{Array address computation}
    \begin{itemize}
        \item Array elements all have the same size: 1, 2, 4 and 8 are common
        \item Suppose an array has elements of size 4 and starts at address {\tt 0x10000}
        \begin{itemize}
            \item The first element (at index 0) is at {\tt 0x10000}
            \item The second element (at index 1) is at {\tt 0x10004}
            \item The third element (at index 2) is at {\tt 0x10008}
            \item Element number {\tt k} is at address {\tt 0x10000 + k*4}
        \end{itemize}
        \item Let's examine the arrays for program ``{\tt array.asm}''
              with {\tt gdb} or {\tt ebe}
    \end{itemize}
    \begin{verbatim}
        segment .bss
a       resb      100
b       resd      100
        align     8
c       resq      100
    \end{verbatim}
\end{frame}

\section{General pattern for memory references}

\begin{frame}[fragile]
    \frametitle{General pattern for memory references}
\small
\begin{tabular}{ll}
 {\tt [label]} & the value contained at label\\
 {\tt [label+2*ind]} & the value contained at the memory address obtained \\
                     & by adding the label and index register times 2\\
 {\tt [label+4*ind]} & the value contained at the memory address obtained \\
                     & by adding the label and index register times 4\\
 {\tt [label+8*ind]} & the value contained at the memory address obtained \\
                     & by adding the label and index register times 8\\
 {\tt [reg]} &   the value contained at the memory address in the register\\
 {\tt [reg+k*ind]} & the value contained at the memory address obtained \\
                     & by adding the register and index register times k\\
 {\tt [label+reg+k*ind]} & the value contained at the memory address obtained \\
                     & by adding the label, the register and index register times k\\
 {\tt [n+reg+k*ind]} & the value contained at the memory address obtained \\
                     & by adding n, the register and index register times k
\end{tabular}
\end{frame}

\begin{frame}
    \frametitle{Memory references}
    \begin{itemize}
        \item For items in the data and bss segments we can use a label
        \item For arrays passed into functions the address is passed in a register
        \item Soon we will be allocating memory using {\tt malloc}
        \begin{itemize}
            \item This address will typically be stored in memory
            \item Later to use the data, we must load the address from memory
                  into a register
            \item Then we can use a register form of memory reference
        \end{itemize}
        \item The use of a number or a label is equivalent to the computer
        \item Both use the same instruction and place the number or label value
              into the same field of the instruction
        \item Using multipliers of 2, 4 or 8 are essentially ``free'' with index
              registers
    \end{itemize}
\end{frame}

\begin{frame}[fragile]
    \frametitle{Example using base registers and an index register}
    \begin{itemize}
        \item In the function below the first parameter is the address of
              the first dword of a destination array
        \item The second parameter is the address of the source array
        \item The third parameter is the number of dwords to copy
        \item It would generally be faster to use ``{\tt rep movsd}''
    \end{itemize}

\begin{verbatim}
        segment .text
        global  copy_array
copy_array:
        xor     ecx, ecx
more:   mov     eax, [rsi+4*rcx]
        mov     [rdi+4*rcx], eax
        add     rcx, 1
        test    rcx, rdx
        jne     more
        xor     eax, eax
        ret
\end{verbatim}
\end{frame}

\section{Allocating arrays}

\begin{frame}[fragile]
    \frametitle{Allocating arrays}
    \begin{itemize}
        \item We will allocate arrays using the C {\tt malloc} function
    \end{itemize}
\begin{verbatim}
    void *malloc ( long size );
\end{verbatim}
    \begin{itemize}
        \item The parameter to {\tt malloc} is the number of bytes to allocate
        \item {\tt malloc} returns the address of the array or 0
        \item Data allocated should be freed, although this will happen when a
              program exits
    \end{itemize}
\begin{verbatim}
        void free ( void *ptr );
\end{verbatim}

\end{frame}

\begin{frame}[fragile]
    \frametitle{Code to allocate an array}
    \begin{itemize}
        \item The code below allocates an array of 1 billion bytes
        \item It saves the pointer to the new array in memory location
              named {\tt pointer}
    \end{itemize}

\begin{verbatim}
        extern  malloc
        ...
        mov     rdi, 1000000000
        call    malloc
        mov     [pointer], rax
\end{verbatim}
\end{frame}

\begin{frame}
    \frametitle{Advantages for using allocated arrays}
    \begin{itemize}
        \item The array will be the right size
        \item There are size limits of about 2 GB in the data and bss segments
        \item The assembler is very slow with large arrays and the program is large
        \item Assembling a program with a 2 GB array in the bss segment
              took about 100 seconds
        \item The executable was over 2 GB
        \item Using {\tt malloc} the program assembles in less than 1 second
              and the program as about 10 KB
        \item Modified to allocate 20 billion bytes the program executes in 3 milliseconds
    \end{itemize}
\end{frame}

\section{Processing arrays}

\begin{frame}
    \frametitle{Processing arrays}
    \begin{itemize}
        \item We present an application which creates an array
        \item Fills the array with random data by calling {\tt random}
        \item Prints the array if the size is small (up to 20 elements)
        \item Determines the minimum value in the array
    \end{itemize}
\end{frame}

\begin{frame}[fragile]
    \frametitle{Creating an array}
    \begin{itemize}
        \item This function allocates an array of double words
        \item The number of double words is the only parameter
        \item Note the use of a stack frame to avoid any problems
              of stack misalignment
    \end{itemize}
\begin{verbatim}
;       array = create ( size );
create:
        push    rbp
        mov     rbp, rsp
        imul    rdi, 4
        call    malloc
        leave
        ret
\end{verbatim}
\end{frame}

\begin{frame}[fragile]
    \frametitle{Filling the array with random numbers}
\footnotesize
\begin{verbatim}
fill:
.array  equ     0
.size   equ     8
.i      equ     16
        push    rbp
        mov     rbp, rsp
        sub     rsp, 32
        mov     [rsp+.array], rdi
        mov     [rsp+.size], rsi
        xor     ecx, ecx
.more   mov     [rsp+.i], rcx
        call    random
        mov     rcx, [rsp+.i]
        mov     rdi, [rsp+.array]
        mov     [rdi+rcx*4], eax
        inc     rcx
        cmp     rcx, [rsp+.size]
        jl      .more
        leave
        ret
\end{verbatim}
\end{frame}

\begin{frame}
    \frametitle{Local labels in {\tt yasm}}
    \begin{itemize}
        \item Labels beginning with a dot are local labels
        \item They are considered part of the previous normal label
        \item The {\tt .more} label could be referenced as {\tt fill.more}
              from outside the {\tt fill} function
        \item The {\tt fill} function keeps saving {\tt rcx} on the stack and
              restoring {\tt rcx} and {\tt rdi} around the {\tt random} call
        \item This could be easier to code using registers which are preserved
              across calls
    \end{itemize}
\end{frame}


\begin{frame}[fragile]
    \frametitle{Filling the array with random numbers (2)}
\scriptsize
\begin{verbatim}
fill:
.r12    equ     0
.r13    equ     8
.r14    equ     16
        push    rbp
        mov     rbp, rsp
        sub     rsp, 32
        mov     [rsp+.r12], r12
        mov     [rsp+.r13], r13
        mov     [rsp+.r14], r14
        mov     r12, rdi        ; r12 is the array address
        mov     r13, rsi        ; r13 is the size
        xor     r14d, r14d      ; loop counter
.more   call    random
        mov     [r12+r14*4], eax
        inc     r14
        cmp     r14, r13
        jl      .more
        mov     r12, [rsp+.r12]
        mov     r13, [rsp+.r13]
        mov     r14, [rsp+.r14]
        leave
        ret
\end{verbatim}
\end{frame}

\begin{frame}[fragile]
    \frametitle{Printing the array}
\footnotesize
\begin{verbatim}
print:
.array  equ     0
.size   equ     8
.i      equ     16
        ...
        segment .data
.format:
        db      "%10d",0x0a,0
        segment .text
.more   lea     rdi, [.format]
        mov     rdx, [rsp+.array]
        mov     rcx, [rsp+.i]
        mov     rsi, [rdx+rcx*4]
        mov     [rsp+.i], rcx
        call    printf
        mov     rcx, [rsp+.i]
        inc     rcx
        mov     [rsp+.i], rcx
        cmp     rcx, [rsp+.size]
        jl      .more
\end{verbatim}
\end{frame}

\begin{frame}[fragile]
    \frametitle{Finding the minimum value in the array}
    \begin{itemize}
        \item This function calls no other function
        \item There is no need for a stack frame
        \item A conditional move is faster than branching
    \end{itemize}

\begin{verbatim}
;       x = min ( a, size );
min:
        mov     eax, [rdi]       ; start with a[0]
        mov     rcx, 1
.more   mov     r8d, [rdi+rcx*4] ; get a[i]
        cmp     r8d, eax
        cmovl   eax, r8d         ; move if smaller
        inc     rcx
        cmp     rcx, rsi
        jl      .more
        ret
\end{verbatim}
\end{frame}

\begin{frame}
    \frametitle{The main program and testing}
    \begin{itemize}
        \item The code is too long, so we will inspect it in an editor
        \item It's also time to test with {\tt gdb} or {\tt ebe}
    \end{itemize}
\end{frame}

\section{Command line parameter array}

\begin{frame}[fragile]
    \frametitle{Command line parameter array}
    \begin{itemize}
        \item The first argument to {\tt main} is the number of command line parameters
        \item The second argument is the address of an array of character pointers, each
              pointing to one of the parameters
        \item Below is a C program illustrating the use of command line parameters
    \end{itemize}
\begin{verbatim}
    #include <stdio.h>

    int main ( int argc, char *argv[] )
    {
        int i;
        for ( i = 0; i < argc; i++ ) {
            printf("%s\n", argv[i]);
        }
        return 0;
    }
\end{verbatim}
\end{frame}

\begin{frame}[fragile]
    \frametitle{Assembly program listing command line parameters}
    \footnotesize
\begin{verbatim}
        segment .data
format  db      "%s",0x0a,0
        segment .text
        global  main        ; let the linker know about main
        extern  printf      ; resolve printf from libc
main:   push    rbp         ; prepare stack frame for main
        mov     rbp, rsp
        sub     rsp, 16
        mov     rcx, rsi    ; move argv to rcx
        mov     rsi, [rcx]  ; get first argv string
start_loop:   
        lea     rdi, [format]
        mov     [rsp], rcx  ; save argv
        call    printf
        mov     rcx, [rsp]  ; restore rsi
        add     rcx, 8      ; advance to next pointer in argv
        mov     rsi, [rcx]  ; get next argv string
        cmp     rsi, 0      
        jnz     start_loop  ; end with NULL pointer
end_loop:
        xor     eax, eax
        leave
        ret
\end{verbatim}
\end{frame}

\end{document}

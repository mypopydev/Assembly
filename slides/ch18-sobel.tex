\documentclass{beamer}

\usepackage{geometry}
\usepackage{graphicx}
%\usepackage{wrapfig}
\usepackage{amsmath}

%\useoutertheme{infolines}
\usetheme{Boadilla}
\usecolortheme{seahorse}
\setbeamertemplate{navigation symbols}{}
\title{Sobel Filter}
\newcommand{\shorttitle}{64 Bit Intel Assembly Language}
\newcommand{\shortauthor}{\copyright 2011 Ray Seyfarth}
\author{Ray Seyfarth}
\begin{document}


\usefoottemplate{\vbox{
\tinycolouredline{structure!55}%
 {\color{white}{\textbf{\shorttitle}\hfill\textbf{\shortauthor}}}%
}}

\begin{frame}
    \titlepage
\end{frame}

\begin{frame}
\frametitle{Outline}
\tableofcontents
\end{frame}

\section{Overview}

\begin{frame}
    \frametitle{Overview}
    \begin{itemize}
        \item The Sobel filter is an image processing edge detection algorithm
        \item It involves convolution of $3\times 3$ image windows with
              2 convolution matrices
    \end{itemize}
$$
S_x = \left[\begin{matrix}
                -1 & 0 & 1 \\
                -2 & 0 & 2 \\
                -1 & 0 & 1
                \end{matrix}\right]
                \qquad\qquad
S_y = \left[\begin{matrix}
                -1 & -2 & -1 \\
                 0 & 0 & 0 \\
                 1 & 2 & 1
                \end{matrix}\right]
$$
    \begin{itemize}
        \item The edge value, $G$, for a pixel at $(r,c)$ is computed by
    \end{itemize}
$$G_x = \sum_{i=-1}^{1}\sum_{j=-1}^1(S_{x,i,j}* I_{r+i,c+i})$$
$$G_y = \sum_{i=-1}^{1}\sum_{j=-1}^1(S_{y,i,j}*I_{r+i,c+i})$$
$$G = \sqrt{G_x^2+G_y^2}$$`
\end{frame}

\section{Sobel in C}

\begin{frame}[fragile]
    \frametitle{A simple C solution}
\small
\begin{verbatim}
#include <math.h>
#define matrix(a,b,c) a[(b)*(cols)+(c)]
void sobel(unsigned char *data, float *output, long rows, long cols)
{
   int r, c;
   int gx, gy;
   for ( r = 1; r < rows-1; r++ ) {
      for ( c = 1; c < cols-1; c++ ) {
         gx = -matrix(data,r-1,c-1) + matrix(data,r-1,c+1) +
              -2*matrix(data,r,c-1) + 2*matrix(data,r,c+1) +
              -matrix(data,r+1,c-1) + matrix(data,r+1,c+1);
         gy = -matrix(data,r-1,c-1) - 2*matrix(data,r-1,c)
              - matrix(data,r-1,c+1) +
              matrix(data,r+1,c-1) + 2*matrix(data,r+1,c)
              + matrix(data,r+1,c+1);
         matrix(output,r,c) = sqrt((float)(gx)*(float)(gx)+
                                   (float)(gy)*(float)(gy));
      }
   }
}
\end{verbatim}
\end{frame}

\section{Sobel computed using SSE instructions}

\begin{frame}[fragile]
    \frametitle{Sobel using SSE instructions}
    \begin{itemize}
        \item 16 8 bit values can be placed in an XMM registers
        \item The central 14 values can be used to compute 14 Sobel results
        \item The code loaded the row $r-1$ and computed part of 14 Sobel results
        \item Then it loaded row $r$ and added more to the 14 Sobel results
        \item Last it loaded row $r+1$ and added more to the 14 Sobel results
        \item The contributions were added, squared, $G_x^2$ added to $G_y^2$ for
              14 $G$ values
        \item The 14 $G$ values were written to the output image
        \item Using 1000 different images it processed 980 images per second
              vs 158 for the C code.
        \item This is 6.2 times as fast
    \end{itemize}
\end{frame}

\begin{frame}
    \frametitle{New instructions used for Sobel}
\begin{description}
    \item [pxor] This instruction performs an exclusive or on a 128 XMM
          source register or memory and stores the result in the destination register.
    \item [movdqa] This instruction moves 128 bits of aligned data from memory
                   to a register, from a register to memory, or from a register
                   to a register.
    \item [movdqu] This instruction moves 128 bits of unaligned data from memory
                   to a register, from a register to memory, or from a register
                   to a register.
    \item [psrldq] This instruction shifts the destination XMM register right the
                   number of bytes specified in the second immediate operand.
    \item [punpcklbw] This instruction unpacks the low 8 bytes of 2 XMM registers
                   and intermingles them. I used this with the second register
                   holding all 0 bytes to form 8 words in the destination.
    \item [punpckhbw] This instruction unpacks the upper 8 bytes of 2 XMM
                   registers and intermingles them.
\end{description}
\end{frame}

\begin{frame}
    \frametitle{New instructions used for Sobel (2)}
\begin{description}
    \item [paddw]  This instruction adds 8 16 bit integers from the second
                   operand to the first operand.  At least one of the operands
                   must be an XMM register and one can be a memory field.
    \item [psubw]  This instruction subtracts the second set of 8 16 bit integers
                   from the first set.
    \item [pmullw] This instruction multiplies the first set of 8 16 bit integers
                   times the second set and store the low order 16 bits of the
                   products in the first operand.
    \item [punpcklwd]   This instruction unpacks and interleaves words from
                   the lower halves 2 XMM registers into the destination register.
    \item [punpckhwd]   This instruction unpacks and interleaves words from
                   the upper halves 2 XMM registers into the destination register.
    \item [cvtdq2ps]  This instruction converts 4 double word integers into
                   4 double word floating point values.
\end{description}
\end{frame}

\begin{frame}
    \frametitle{The Sobel assembly code}
    \begin{itemize}
        \item This code is far too long to examine in slides
    \end{itemize}
\end{frame}

\end{document}

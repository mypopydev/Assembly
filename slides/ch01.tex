\documentclass{beamer}

\usepackage{geometry}
\usepackage{graphicx}
%\usepackage{wrapfig}
\usepackage{amsmath}

%\useoutertheme{infolines}
\usetheme{Boadilla}
\usecolortheme{seahorse}
\setbeamertemplate{navigation symbols}{}
\title{Introduction to 64 Bit Intel Assembly Language Programming}
\newcommand{\shorttitle}{64 Bit Intel Assembly Language}
\newcommand{\shortauthor}{\copyright 2011 Ray Seyfarth}
\author{Ray Seyfarth}
\begin{document}


\usefoottemplate{\vbox{
\tinycolouredline{structure!55}%
 {\color{white}{\textbf{\shorttitle}\hfill\textbf{\shortauthor}}}%
}}

\begin{frame}
    \titlepage
\end{frame}
\begin{frame}
\frametitle{Outline}
\tableofcontents
\end{frame}

\section{Goals for this course}

\begin{frame}
    \frametitle{Goals for this course}
    \begin{itemize}
        \item Learn internal data formats
        \item Learn basic 64 bit Intel/AMD instructions
        \item Write pure assembly programs
        \item Write mixed C and assembly programs
        \item Use the {\tt gdb} debugger
        \item Floating point instructions
        \item Arrays
        \item Functions
        \item Structs
        \item Data structures
        \item Using system calls and C libraries
        \item SSE and AVX instructions
    \end{itemize}

\end{frame}

\section{Why study assembly language?}

\begin{frame}
    \frametitle{Problems with assembly language}
    \begin{itemize}
        \item Assembly is the poster child for non-portability
        \begin{itemize}
            \item Different CPU = different assembly
            \item Different OS = different function ABI
            \item Intel/AMD CPUs operate in 16, 32 and 64 bit modes
        \end{itemize}
        \item Difficult to program
        \begin{itemize}
            \item More time = more money
            \item Less reliable
            \item Difficult to maintain
        \end{itemize}
        \item Syntax does not resemble mathematics
        \item No syntactic protection
            \begin{itemize}
                \item No structured ifs, loops
            \end{itemize}
        \item No typed variables
            \begin{itemize}
                \item Can use a pointer as a floating point number
                \item Can load a 4 byte integer from a double variable
            \end{itemize}
        \item Variable access is roughly like using pointers
        \item Language is not orthogonal
    \end{itemize}
\end{frame}

\begin{frame}
    \frametitle{What's good about assembly language?}
    \begin{itemize}
        \item Assembly language is fast
        \begin{itemize}
            \item Optimizing C/C++ compilers can be faster
            \item You need to dissect an algorithm and rearrange it to use
                  a special feature that the compiler can't figure out
            \item Generally you must use a special instructions
            \item There are over 1000 instructions
            \item Still it can be faster
        \end{itemize}
        \item Assembly programs are small
        \begin{itemize}
            \item But memory is cheap and plentiful
            \item C/C++ compilers can optimize for size
            \item Compilers can re-order code sections to reduce size
        \end{itemize}
        \item Assembly can do things not possible in C/C++
        \begin{itemize}
            \item I/O instructions
            \item Manage memory mapping registers
            \item Manipulate other internal control registers
        \end{itemize}

    \end{itemize}
\end{frame}

\begin{frame}
    \frametitle{What's good about assembly for ordinary mortals?}
    \begin{itemize}
        \item Explains how the computer works
        \item Numbers are stored in registers
        \item Arithmetic is done with registers
        \item C function register and stack usage defined
        \item Stack frames are used by debuggers
        \item Optimization techniques are explained
        \item Computer bugs are more immediately related to machine
              instructions and limitations
        \item You will learn how the compiler implements
            \begin{itemize}
                \item if/else statements
                \item loops
                \item functions
                \item structures
                \item arrays
                \item recursion
            \end{itemize}
        \item Your C/C++ coding will improve
    \end{itemize}
\end{frame}

\section{What is a computer?}

\begin{frame}
    \frametitle{What is a computer?}
    \begin{itemize}
        \item A machine to process bits
        \begin{itemize}
            \item We consider the bits to mean things
            \item True or false
            \item Integers
            \item Floating point numbers
            \item Characters and strings
            \item User-created types
            \begin{itemize}
                \item Physical objects, animals, plants, minerals
                \item Lists of things
                \item Stacks of things
                \item Queues of things
                \item Priority queues of things
                \item Trees of various types
                \item Hash tables
            \end{itemize}
        \end{itemize}
    \end{itemize}
\end{frame}

\begin{frame}
    \frametitle{Bytes}
    \begin{itemize}
        \item Memory is organized as 8 bit bytes
        \item First byte of memory is at address 0
        \item Second byte is at address 1
        \item Memory is an array of bytes
        \item Consider the byte with bits {\tt 01010101}
        \begin{itemize}
            \item Considered as a decimal number it is 85
            \item In the right context it can be a machine instruction
            \begin{itemize}
                \item Push the {\tt rbp} register onto the run-time stack
            \end{itemize}
            \item Considered as a character is it '{\tt U}'
            \item It could be part of the string ``{\tt Undefined}''
            \item It could be part of a larger number, like 85*256+17 = 21777
            \item It could be part if an address in the computer
        \end{itemize}
    \end{itemize}
\end{frame}

\begin{frame}
    \frametitle{Program execution}
    \begin{columns}[c]
        \column{.05\textwidth}
            \quad
        \column{.5\textwidth}
            The 12 bytes to the right constitute a program which
            if placed in memory and executed, simply exits with status 5.
            The addresses are shown in hexadecimal to emphasize that the addresses
            are fairly close to the beginning of a page starting at {\tt 0x400000}.
        \column{.05\textwidth}
            \quad
        \column{.4\textwidth}
            \begin{tabular}{|c|c|}
            \hline
            Address & Value \\
            \hline
            4000b0 & 184 \\
            \hline
            4000b1 & 1 \\
            \hline
            4000b2 & 0 \\
            \hline
            4000b3 & 0 \\
            \hline
            4000b4 & 0 \\
            \hline
            4000b5 & 187 \\
            \hline
            4000b6 & 5 \\
            \hline
            4000b7 & 0 \\
            \hline
            4000b8 & 0 \\
            \hline
            4000b9 & 0 \\
            \hline
            4000ba & 205 \\
            \hline
            4000bb & 128 \\
            \hline
            \end{tabular}
    \end{columns}
\end{frame}

\section{Machine language}

\begin{frame}
    \frametitle{Machine language}
    \begin{itemize}
        \item Machine language is a sequence of bytes
        \item The bytes specify instructions and data
        \item Many instructions include a data address
        \item Branching instructions include instruction addresses
        \item Adding a new instruction changes all subsequent instruction addresses
        \item Changes to data can alter data addresses
            \begin{itemize}
                \item Increasing an array size changes all subsequent data addresses
                \item Adding a data item can change subsequent data addresses
            \end{itemize}
        \item Each changed address must be changed in all instructions using the address
        \item This is hard enough to stimulate creativity
        \item People figured out how to use symbolic names for data and instructions
    \end{itemize}
\end{frame}

\section{Assembly language}

\begin{frame}
    \frametitle{Second generation language}
    \begin{itemize}
        \item First generation - machine language
        \item Second generation - assembly language
        \begin{itemize}
            \item Names for instructions
            \item Names for variables
            \item Names for locations of instructions
            \item Perhaps with macros - code replacement
        \end{itemize}
        \item Third generation - not machine instructions
        \begin{itemize}
            \item Modeled after mathematics - Fortran
            \item Modeled after English - Cobol
            \item List processing - Lisp
        \end{itemize}
        \item Fourth generation - domain specific
        \begin{itemize}
            \item SQL
        \end{itemize}
        \item Fifth generation - describe problem, computer generates algorithm
        \begin{itemize}
            \item Prolog
        \end{itemize}
    \end{itemize}
\end{frame}

\begin{frame}[fragile]
\frametitle{Assembly example}

\begin{verbatim}
;   Program: exit
;
;   Executes the exit system call
;
;   No input
;
;   Output: only the exit status ($? in the shell)
;
    segment .text
    global  _start
_start:
    mov  eax,1       ; 1 is the exit syscall number
    mov  ebx,5       ; the status value to return
    int  0x80        ; execute a system call
\end{verbatim}
\end{frame}

\begin{frame}
    \frametitle{Assembly syntax}
    \begin{itemize}
        \item {\tt ;} starts comments
        \item Labels are strings which are not instructions
        \begin{itemize}
            \item Usually start in column 1
            \item Can end with a colon to avoid confusion with instructions
        \end{itemize}
        \item Instructions can be machine instructions or assembler instructions
        \begin{itemize}
            \item {\tt mov} and {\tt int} are machine instructions or opcodes
            \item {\tt segment} and {\tt global} are assembler instructions
                  or pseudo-ops
        \end{itemize}
        \item Instructions can have operands
        \begin{itemize}
            \item {\tt here:  mov  eax, 1}
            \item {\tt here} is a label for the instruction
            \item {\tt mov} is an opcode
            \item {\tt eax} and {\tt 1} are operands
        \end{itemize}
    \end{itemize}
\end{frame}

\begin{frame}
    \frametitle{Some assembler instructions}
    \begin{itemize}
        \item {\tt section} or {\tt segment} define a part of the program
        \begin{itemize}
            \item {\tt .text} is where instructions go for Linux
        \end{itemize}
        \item {\tt global} defines a label to be used by the linker
        \item {\tt global \_start} makes {\tt \_start} a global label
        \item {\tt \_start} or {\tt main} is where a program starts
        \begin{itemize}
            \item {\tt \_start} is more basic
            \item {\tt main} is called (perhaps indirectly) by {\tt \_start}
        \end{itemize}
    \end{itemize}
\end{frame}

\section{Assembling and linking}

\begin{frame}
    \frametitle{Assembling the exit program}
    \begin{itemize}
        \item {\tt yasm -f elf64 -g dwarf2 -l exit.lst exit.asm}
        \item {\tt -f elf64} says we want a 64 bit object file
        \item {\tt -g dwarf2} says we want {\tt dwarf2} debugging info
        \begin{itemize}
            \item {\tt dwarf2} works pretty well with the {\tt gdb} debugger
        \end{itemize}
        \item {\tt -l exit.lst} asks for a listing in {\tt exit.lst}
        \item {\tt yasm} will produce {\tt exit.o}, an object file
        \begin{itemize}
            \item machine instructions not ready to execute
        \end{itemize}
    \end{itemize}
\end{frame}

\begin{frame}[fragile]
\frametitle{\tt exit.lst}
\begin{verbatim}
     1                                 %line 1+1 exit.asm
     2                                 
     3                                 
     4                                 
     5                                 
     6                                 
     7                                 
     8                                 
     9                                 
    10                                 [segment .text]
    11                                 [global _start]
    12                                 
    13                                 _start:
    14 00000000 B801000000              mov eax,1
    15 00000005 BB05000000              mov ebx,5
    16 0000000A CD80                    int 0x80
\end{verbatim}
\end{frame}

\begin{frame}
    \frametitle{Linking}
    \begin{itemize}
        \item Linking means combining object files to make an executable file
        \item For programs with {\tt \_start}
        \begin{itemize}
            \item {\tt ld -o exit exit.o}
            \item Builds a file named {\tt exit}
            \item Default is {\tt a.out}
        \end{itemize}
        \item For programs with {\tt main}
        \begin{itemize}
            \item {\tt gcc -o exit exit.o}
            \item Gets default {\tt \_start} function from the C library
        \end{itemize}
        \item {\tt ./exit} to run the program
    \end{itemize}
\end{frame}


\begin{frame}
    \frametitle{Assembling, linking and running with ebe}
    \begin{itemize}
        \item ebe is designed to support assembly programming
        \item Click on a line number to set a breakpoint
        \item Run button in ebe
        \begin{itemize}
            \item assembles with yasm
            \item links with gcc or ld
            \item executes the program with gdb
        \end{itemize}
        \item Program stops at breakpoint
        \item Next button to single-step
    \end{itemize}
\end{frame}

\end{document}

\documentclass{beamer}

\usepackage{geometry}
\usepackage{graphicx}
%\usepackage{wrapfig}
\usepackage{amsmath}

%\useoutertheme{infolines}
\usetheme{Boadilla}
\usecolortheme{seahorse}
\setbeamertemplate{navigation symbols}{}
\title{Structs}
\newcommand{\shorttitle}{64 Bit Intel Assembly Language}
\newcommand{\shortauthor}{\copyright 2011 Ray Seyfarth}
\author{Ray Seyfarth}
\begin{document}


\usefoottemplate{\vbox{
\tinycolouredline{structure!55}%
 {\color{white}{\textbf{\shorttitle}\hfill\textbf{\shortauthor}}}%
}}

\begin{frame}
    \titlepage
\end{frame}

\begin{frame}[fragile]
    \frametitle{Structs in C}
    \begin{itemize}
        \item A struct is a compound object
    \end{itemize}
\begin{verbatim}
    struct Customer {
        int  id;
        char name[64];
        char address[64];
        int  balance;
    };
\end{verbatim}
    \begin{itemize}
        \item We can allocate a {\tt Customer} if we know the size
    \end{itemize}
\begin{verbatim}
    mov     rdi, 136        ; size of a Customer
    call    malloc
    mov     [c], rax        ; save the address
\end{verbatim}
\end{frame}

\begin{frame}[fragile]
    \frametitle{Filling in a C struct}
    \begin{itemize}
        \item We can fill in the parts given their offsets
    \end{itemize}
\begin{verbatim}
        mov     [rax], dword 7  ; set the id
        lea     rdi, [rax+4]    ; name field
        lea     rsi, [name]     ; name to copy to struct
        call    strcpy
        mov     rax, [c]
        lea     rdi, [rax+68]   ; address field
        lea     rsi, [address]  ; address to copy
        call    strcpy
        mov     rax, [c]
        mov     edx, [balance]
        mov     [rax+132], edx
\end{verbatim}
\end{frame}

%\begin{frame}
%\frametitle{Outline}
%\tableofcontents
%\end{frame}

\begin{frame}[fragile]
    \frametitle{Assembly struct}
    \begin{itemize}
        \item Using the {\tt yasm} {\tt struc} pseudo-op we can
              define a Customer
    \end{itemize}
\begin{verbatim}
        struc   Customer
id      resd    1
name    resb    64
address resb    64
balance resd    1
        endstruc
\end{verbatim}
    \begin{itemize}
        \item {\tt id}, {\tt name}, {\tt address} and {\tt balance} are
              globals
        \item You could not have {\tt id} in 2 structs
        \item It's almost the same as doing 4 equates
        \item The size is {\tt Customer\_size}
    \end{itemize}
\end{frame}

\begin{frame}[fragile]
    \frametitle{Assembly struct}
    \begin{itemize}
        \item One alternative is to prefix field names with dots
    \end{itemize}
\begin{verbatim}
         struc   Customer
.id      resd    1
.name    resb    64
.address resb    64
.balance resd    1
         endstruc
\end{verbatim}
    \begin{itemize}
        \item Then you would have to use {\tt Customer.id}
        \item Another alternative is to use an abbreviated prefix
    \end{itemize}
\begin{verbatim}
        struc   Customer
c_id      resd    1
c_name    resb    64
c_address resb    64
c_balance resd    1
        endstruc
\end{verbatim}
\end{frame}

\begin{frame}[fragile]
    \frametitle{Program to allocate and fill a struct - data segment}
\begin{verbatim}
        segment .data
name    db      "Calvin", 0
address db      "12 Mockingbird Lane",0
balance dd      12500
        struc   Customer
c_id      resd    1
c_name    resb    64
c_address resb    64
c_balance resd    1
        endstruc
c       dq      0       ; to hold a Customer pointer
\end{verbatim}
\end{frame}

\begin{frame}[fragile]
    \frametitle{Program to allocate and fill a struct - part of text segment}
\begin{verbatim}
        mov     rdi, Customer_size
        call    malloc
        mov     [c], rax    ; save the pointer
        mov     [rax+c_id], dword 7
        lea     rdi, [rax+c_name]
        lea     rsi, [name]
        call    strcpy
        mov     rax, [c]    ; restore the pointer
        lea     rdi, [rax+c_address]
        lea     rsi, [address]
        call    strcpy
        mov     rax, [c]    ; restore the pointer
        mov     edx, [balance]
        mov     [rax+c_balance], edx
        xor     eax, eax
\end{verbatim}
\end{frame}

\begin{frame}[fragile]
    \frametitle{Alignment problems}
    \begin{itemize}
        \item Suppose you increase the size of the {\tt c\_address} array to 65
        \item C would make the offset of {\tt balance} be 136
        \item {\tt yasm} would define the offset as 133
        \item The goal is to be C compatible
        \item Also C would have {\tt sizeof(Customer)} as 140
        \item {\tt Customer\_size} would be 137
        \item C aligns each field and makes the size of a struct appropriate
              for aligning each data item properly if we allocate an array of structs
        \item We need to use {\tt align} in the struct
    \end{itemize}
\begin{verbatim}
        struc   Customer
c_id      resd    1
c_name    resb    64
c_address resb    65
          align   4
c_balance resd    1
        endstruc
\end{verbatim}
\end{frame}

\begin{frame}[fragile]
    \frametitle{Allocating a slightly more complex array of customers}
\begin{verbatim}
          segment .data
          struc   Customer
c_id      resd    1     ; 4 bytes
c_name    resb    65    ; 69 bytes
c_address resb    65    ; 134 bytes
          align   4     ; aligns to 136
c_balance resd    1     ; 140 bytes
c_rank    resb    1     ; 141 bytes
          align   4     ; aligns to 144
          endstruc
customers dq      0
          segment .text
          mov   edi, 100 ; for 100 structs
          mul   edi, Customer_size 
          call  malloc
          mov   [customers], rax
\end{verbatim}
\end{frame}

\begin{frame}[fragile]
    \frametitle{Printing an array of customers}
\small
\begin{verbatim}
          segment .data
format    db    "%s %s %d",0x0a,0
          segment .text
          push  r15
          push  r14
          mov   r15, 100        ; counter saved through calls
          mov   r14, [customers]; pointer saved through calls
more      lea   edi, [format]
          lea   esi, [r14+c_name]
          lea   rdx, [r14+c_address]
          mov   rcx. [r14+c_balance]
          call  printf
          add   r14, Customer_size
          sub   r15, 1
          jnz   more
          pop   r14
          pop   r15
          ret
\end{verbatim}
\end{frame}

\end{document}
